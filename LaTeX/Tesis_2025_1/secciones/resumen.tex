\customchapter{RESUMEN}

Este informe presenta el diseño e implementación de un banco de pruebas oscilante de 1GDL, motivado por la importancia de las pruebas de oscilación mecánica para optimizar el diseño y garantizar la integridad operativa de los sistemas, especialmente en la ingeniería mecatrónica.  El documento destaca la dificultad de predecir la respuesta de componentes y sistemas a cargas oscilatorias sin pruebas experimentales, lo que puede resultar en diseños deficientes y fallos prematuros.  

El banco de pruebas propuesto busca validar diseños y diagnosticar fallos en prototipos o componentes, sometiéndolos a condiciones oscilatorias controladas y representativas.  Se justifica su desarrollo por la necesidad de comprender y mitigar los efectos de las oscilaciones en sistemas mecatrónicos, mejorar su fiabilidad y seguridad, y abrir la puerta a futuras investigaciones en el control de oscilaciones. \\


\noindent \textbf{Palabras clave:}\\
\noindent Banco de pruebas oscilante; Sistemas mecatrónicos; Validación experimental; Oscilaciones mecánicas; Acelerómetros; Respuesta dinámica; Excitación mecánica controlada